%===============================================================================
% $Id: ifacconf.tex 19 2011-10-27 09:32:13Z jpuente $  
% Template for IFAC meeting papers
% Copyright (c) 2007-2008 International Federation of Automatic Control
%===============================================================================
\documentclass{ifacconf}

\usepackage{graphicx}      % include this line if your document contains figures
\usepackage{natbib}        % required for bibliography

\usepackage{amssymb,mathrsfs, amsmath}
\usepackage{tikz}
\usepackage{tikz-cd}
\usepackage{multicol}

%newcommand
\newcommand\g[1]{\textbf{#1}}
\newcommand\id[1]{\text{id}_{#1}}
\newcommand\sr{\text{sr}(D)}
\newcommand\im{\text{im}}
\newcommand\p{\text{pr}}
%===============================================================================
\begin{document}
\begin{frontmatter}

\title{A constructive version of Warfield's Theorem} 
% Title, preferably not more than 10 words.


\author[First]{Cyrille Chenavier} 
\author[Second]{Alban Quadrat} 


\address[First]{Inria Lille - Nord Europe,
  Villeneuve d'Ascq, France (e-mail: cyrille.chenavier@inria.fr).}
\address[Second]{Inria Paris, Universit\'e Pierre et Marie Curie,  
   Paris, France (e-mail: alban.quadrat@inria.fr).}

\begin{abstract}                % Abstract of not more than 250 words.
\end{abstract}

\begin{keyword}

\end{keyword}

\end{frontmatter}
%===============================================================================

\section{Introduction}


\section{Isomorphisms and equivalent matrices}

In this section, we recall the characterization of morphisms between
finitely presented left $D$-modules, as well as results of Fitting and
Warfield which rely isomorphic left $D$-modules to matrix conjugation.

\subsection{Effective version of Fitting's Theorem}

Consider two left $D$-modules $M$ and $M'$ with finite presentations:
\medskip
\begin{small}
  \[\begin{tikzcd}
  D^{1\times q}\arrow[r, ".R"] &
  D^{1\times p}\arrow[r, "\pi"] &
  M\arrow[r] &
  0, 
  \\
  D^{1\times q'}\arrow[r, ".R'"] &
  D^{1\times p'}\arrow[r, "\pi'"] &
  M'\arrow[r] &
  0, 
  \end{tikzcd}\]
  \end{small}
\medskip
namely, \emph{exact sequences} (see~\cite{Ro:09}), where
$R\in~ D^{q\times p}$, $(.R)(\mu)=\mu R$, for every $\mu\in D^{1\times q}$
and $\pi$ is the natural projection on $M=D^{1\times p}/(D^{1\times q}R)$
(similarly for $R'$ and $\pi'$).

From~\cite{Ro:09}, there exists $f\in\hom_D(M,M')$ if and only if there
exist matrices $P\in D^{p\times p'}$ and $Q\in D^{q\times q'}$ such that
$RP=QR'$ and
\medskip
\[\forall\lambda\in D^{1\times p},\ f(\pi(\lambda))=\pi'(\lambda\pi).\]
\medskip
Hence, the following diagram is exact and commutative:
\medskip
\begin{center}
  \begin{small}
    \begin{tikzcd}
      D^{1\times q}\arrow[r, ".R"] \arrow[d, ".Q"'] &
      D^{1\times p}\arrow[r, "\pi"] \arrow[d, ".P"']&
      M\arrow[r] \arrow[d, "f"']&
      0
      \\
      D^{1\times q'}\arrow[r, ".R'"] &
      D^{1\times p'}\arrow[r, "\pi'"] &
      M'\arrow[r] &
      0
    \end{tikzcd}
  \end{small}
\end{center}
\medskip

We let $n:=q+p'+p+q'$ and $m:=p+p'$. The two $n\times m$ matrices
\medskip
\begin{equation}\label{equ:def_of_L}
  \begin{small}
    L:=\begin{pmatrix}
    R & 0 \\
    0 & \id{p'} \\
    0 & 0 \\
    0 & 0
    \end{pmatrix}\ \ \text{and}\ \
    L':=\begin{pmatrix}
    0 & 0 \\
    0 & 0 \\
    \id{p} & 0 \\
    0 & R'
    \end{pmatrix},
  \end{small}
\end{equation}
\medskip
induce finite presentations: $M\simeq D^{1\times m}/(D^{1\times n}L)$ 
and $M'\simeq D^{1\times m}/(D^{1\times n}L')$. In~\cite{ClQu:11}, an
effective version of a result due to~\cite{Fi:36} is given. If $f$ is an
isomorphism, then $L$ and $L'$ are equivalent: there exist $6$ matrices
$R_2\in D^{r\times q}$, $R'_2\in D^{r'\times q'}$,
$Z_2\in D^{p\times q}$, $Z_2\in D^{q\times r}$,
$Z'_2\in D^{q'\times r'}$, $Z\in D^{p\times q}$ and
$Z'\in D^{p'\times q'}$ and two invertible matrices of size $m$ and $n$
\medskip
\begin{equation}\label{equ:def_of_XF_YF}
  \begin{small}
    X_F:=\begin{pmatrix}
    \id{p} & P \\
    -P' & \id{p'}-P'P
  \end{pmatrix}\ \ \text{and}\ \
    Y_F:=\begin{pmatrix}
    \id{q} & 0 & R & Q \\
    0 & \id{p'} & -P' & Z' \\
    -Z & P & 0 & PZ'-ZQ \\
    -Q' & -R' & 0 & Z'_2R'_2
    \end{pmatrix},
  \end{small}
\end{equation}
\medskip
with inverses
\medskip
\begin{equation}\label{equ:def_of_XF-_YF-}
  \begin{small}
    X_F^-:=\begin{pmatrix}
    \id{p}-PP' & -P \\
    P' & \id{p'}
  \end{pmatrix}\ \ \text{and}\ \
    Y_F^-:=\begin{pmatrix}
    Z_2R_2& 0 & -R & -Q \\
    P'Z-Z'Q' & 0 & P' & -Z' \\
    Z & -P & \id{p} & 0 \\
    Q' & R' & 0 & \id{q'}
    \end{pmatrix},
  \end{small}
\end{equation}
such that
\begin{small}
  \begin{equation}\label{equ:Fitting_eq}
    L'=Y_F^-LX_F.
  \end{equation}
\end{small}
\medskip
In other words, the following diagram is exact commutative
\begin{center}
    \begin{tikzcd}
      D^{1\times n}\arrow[r, ".L"] \arrow[d, ".Y_W"', shift right=1ex] &
      D^{1\times m}\arrow[r, "\pi\oplus 0"] \arrow[d, ".X_W"', shift right=1ex]&
      M\arrow[r] \arrow[d, "f"', shift right=1ex]&
      0
      \\
      D^{1\times n}\arrow[r, ".L'"] \arrow[u, ".Y_W^-"', shift right=1ex]&
      D^{1\times m}\arrow[r, "0\oplus\pi'"] \arrow[u, ".X_W^-"', shift right=1ex]&
      M'\arrow[r] \arrow[u, "f^-"', shift right=1ex]&
      0
    \end{tikzcd}
  \end{center}



\vspace{-0.4cm}

\subsection{Warfield's Theorem}

A result due to~\cite{Wa:78} asserts that that the size of $0$ an id
blocs in (\ref{equ:def_of_L}) can be reduced, whereas the new matrices
are still equivalent. This result is based on the notion of
\emph{stable rank}. The definition of the latter requires to introduce
various notions that we present now.

A column vector $u:=(u_1\cdots u_k)^T\in D^{k\times 1}$ is called
\emph{unimodular} if there exists a line $v\in D^{1\times k}$ such that
$vu=1$. Moreover, $u$ is said to be \emph{stable} if there exist 
$d_1,\cdots,d_{k-1}\in D$ such that
$(u_1+d_1u_k\cdots u_{k-1}+d_{k-1}u_k)$ is unimodular. An integer $r$ is
said to be in the stable rank of $D$ if whenever $k>r$, every column
$u\in D^{k\times 1}$ is stable. The stable rank $\sr$ of $D$ is the
smallest integer in the stable rank of $D$.

Assume that the two matrices (\ref{equ:def_of_L}) are equivalent, then
Warfield's Theorem asserts that if there exist two integers $r$ and $s$
such that
\medskip
\begin{equation}\label{equ:s_r}
  \left\{
  \begin{split}
    & s\leq\min(p+q',q+p'),\\
    & \sr\leq\max(p+q'-s,q+p'-s),\\
    &r\leq\min(p,p'), \\
    & \sr\leq\max(p-r, p'-r),
  \end{split}
  \right.
\end{equation}
\medskip
then the following $(n-r-s)\times (m-r)$ matrices are equivalent
\medskip
\begin{equation}\label{equ:def_of_L_bar}
  \begin{small}
    \overline{L}:=\begin{pmatrix}
    R & 0 \\
    0 & \id{p'-r} \\
    0 & 0
    \end{pmatrix}\ \ \text{and}\ \
    \overline{L'}:=\begin{pmatrix}
    0 & 0 \\
    \id{p-r} & 0 \\
    0 & R'
    \end{pmatrix},
  \end{small}
\end{equation}
\medskip
and induce finite presentations of $M$ and $M'$, respectively.

In the next section, we introduce a procedure which computes invertible
matrices $X_W$ and $Y_W$ such that
\medskip
\[\overline{L'}=Y_W^-\overline{L}X_W.\]

\section{Effective Warfield's Theorem}

Throughout this Section, we fix some notations. Let $M$ and $M'$ be two
left $D$-modules, isomorphic with $f:M\overset{\sim}{\to}M'$, finitely
presented by matrices $R\in D^{q\times p}$ and $R'\in D^{q'\times p'}$,
respectively, and let $L$, $L'$, $X_F$, $Y_F$, $X_F^-$ and $Y_f^-$ be the
matrices defined in (\ref{equ:def_of_L}) (\ref{equ:def_of_XF_YF}) and
(\ref{equ:def_of_XF-_YF-}). Let $m:=q+p'+p+q'$ and $n:=p+p'$. Given a
nonzero integer $k$, we let $\overline{k}:=k-1$. For $1\leq i\leq k$, the
$i$-th vector of the canonical basis of $D^{1\times k}$ is written $e_i^k$. 

\subsection{Reduction of the zero bloc}

In this section, we present the procedure for removing one $0$ in $L$ and
$L'$. Inductive applications of this procedure enables us to remove many
$0$. Without lost of generalities, we may suppose that $q+p'\leq p+q'$,
and we assume that $\sr\leq\overline{p+q'}$, so that the hypotheses of
(\ref{equ:s_r}) are fullfilled for $s=1$. Our purpose is to show that the
following $\overline{n}\times m$ matrices are equivalent:
\medskip
\begin{small}
  \[\tilde{L}:=\begin{pmatrix}
  R & 0\\
  0 & \id{p'}\\
  0 & 0
  \end{pmatrix}\ \ \text{and}\ \
  \tilde{L}':=\begin{pmatrix}
  0 & 0\\
  \id{p} & 0\\
  0 & R'
  \end{pmatrix}.\]
\end{small}

\smallskip

\begin{prop}\label{prop:reduced_Bezout_zero_part}
  There exist $\g{c},\ \g{u}\in D^{1\times p}$ and
  $\g{d},\ \g{v}\in D^{1\times\overline{q'}}$ such that
  \medskip
  \begin{equation}\label{equ:reduced_Bezout_zero_part}
    \begin{small}
      \begin{pmatrix}
        0 & \g{c} & \g{d}\end{pmatrix}\begin{pmatrix}
        \id{q+p'} & 0 & 0 & 0\\
        0 & \id{p} & 0 & \g{u}^T\\
        0 & 0 & \id{\overline{q'}} & \g{v}^T
      \end{pmatrix}Y_F(e^{n}_{q+p'})^T=1.
    \end{small}
  \end{equation}

\end{prop}

\begin{pf}
  Getting terms of the $p'$-th column and $p'$-th line in the relation
  $\id{p'}=P'P+Z'R'$, we get the following:
  \medskip
  \begin{small}
    \[\sum_{k=1}^pP'_{p'k}P_{kp'}+\sum_{k=1}^{q'}Z'_{p'k}R'_{kp'}=1.\]
  \end{small}
  \medskip
  From $\sr\leq\overline{p+q'}$, we deduce that there exist $c_1,
  \cdots,\ c_p,\ d_1,\cdots,\ d_{\overline{q'}},\ u_1,\cdots,\ u_p,\ v_1,
  \cdots,\ v_{\overline{q'}}\in D$ such that
  \medskip
  \begin{small}
    \begin{equation}\label{equ:reduced_Bezout_zero_part_proof}
      \sum_{k=1}^pc_k\left(P_{kp'}+u_kR'_{q'p'}\right)+\sum_{k=1}^{
        \overline{q'}}d_k\left(R'_{kp'}+v_kR'_{q'p'}\right)=1.
    \end{equation}
  \end{small}
  \medskip
  Letting $\g{c}:=\left(c_1,\cdots,\ c_p\right)$, $\g{d}:=\left(-d_1,
  \cdots,\ -d_{\overline{q'}}\right)$, $\g{u}:=~\left(-u_1,\cdots,\ -u_p
  \right)$ and $\g{v}:=\left(v_1,\cdots,\ v_{\overline{q'}}\right)$, the
  hand side of \eqref{equ:reduced_Bezout_zero_part_proof} is the left
  hand side of \eqref{equ:reduced_Bezout_zero_part}, which proves
  Proposition~\ref{prop:reduced_Bezout_zero_part}.
\end{pf}

\medskip

With the notations of Proposition~\ref{prop:reduced_Bezout_zero_part},
we introduce the lines $\tilde{\ell}\in D^{1\times\overline{n}}$ and 
$\ell\in D^{1\times n}$ defined as follows:
\medskip
\[\tilde{\ell}:=\begin{pmatrix}
  0 & \g{c} & \g{d}\end{pmatrix}\ \ \text{and}\ \
  \ell:=\begin{pmatrix}
  0 & \g{c} & \g{d} & 0
    \end{pmatrix},
\]
\medskip
as well as the matrices $U\in D^{n\times n}$,
$F\in D^{\overline{n}\times n}$ defined as follows:
\medskip
\begin{small}
  \[U:=\begin{pmatrix}
    \id{q+p'} & 0 & 0 & 0\\
    0 & \id{p} & 0 & \g{u}^T\\
    0 & 0 & \id{\overline{q'}} & \g{v}^T\\
    0 & 0 & 0 & 1
    \end{pmatrix}\ \ \text{and}\ \
    F:=\begin{pmatrix}
    \id{\overline{n}} & 0
    \end{pmatrix}UY_F.
  \]
\end{small}
\medskip

From Relation (\ref{equ:reduced_Bezout_zero_part}) and
\medskip
\begin{small}
  \[F=\begin{pmatrix}
  \id{q+p'} & 0 & 0 & 0\\
  0 & \id{p} & 0 & \g{u}^T\\
  0 & 0 & \id{\overline{q'}} & \g{v}^T
  \end{pmatrix}Y_F,
  \]
\end{small}
\medskip
we get:
\medskip
\begin{equation}\label{equ:extend_Bezout_zero_part}
  1=\tilde{\ell}F(e^n_{q+p'})^T=\ell UY_F(e^n_{q+p'})^T.
\end{equation}

We consider the matrices $\p,\ \p'\in D^{n\times\overline{n}}$
\begin{small}
  \[\begin{split}
  \p:=&\begin{pmatrix}
  \id{\overline{n}}-F(e^{n}_{q+p'})^T\tilde{\ell}\\
  \tilde{\ell}
  \end{pmatrix},\\
  \p':=&\left(\id{n}-(e^{n}_{q+p'})^T\ell UY_F\right)\begin{pmatrix}
  \id{q+\overline{p'}} & 0\\
  0 & 0\\
  0 & \id{p+q'}
  \end{pmatrix},
  \end{split}\]
  \end{small}
\medskip
and $\iota,\ \iota'\in D^{\overline{n}\times n}$ 
\begin{small}
  \[
  \iota:=\begin{pmatrix}
  \id{\overline{n}}-F(e^{n}_{q+p'})^T\tilde{\ell} &
  F(e^{n}_{q+p'})^T
  \end{pmatrix},\
  \iota':=\begin{pmatrix}
  \id{q+\overline{p'}} & 0 & 0\\
  0 & 0 & \id{p+q'}
  \end{pmatrix}.\]
\end{small}
\medskip

\begin{prop}\label{prop:exact_sequences_zero_part}
  We have the following relations:
  \vspace{-0.5cm}
  \begin{multicols}{2}
    \begin{enumerate}
    \item\label{it:split_pi_zero_part} $\iota\p=\id{\overline{n}}$,
    \item\label{it:split_pip_zero_part} $\iota'\p'=\id{\overline{n}}$,
    \item\label{it:ker_pi_zero_part} $\ker(.\p)=\emph{D}\ell$,
    \item\label{it:ker_pip_zero_part} $\ker(.\p')=\emph{D}\ell UY_F$.
    \end{enumerate}
  \end{multicols}
\end{prop}

\begin{pf}
  \begin{enumerate}
  \item From \eqref{equ:extend_Bezout_zero_part}, we have
    $\left(F(e^n_{q+p'})^T\tilde{\ell}\right)^2=F(e^{n}_{q+p'})^T
    \tilde{\ell}$, from which we deduce $\iota\p=\id{\overline{n}}$ by
    computing the matrix product.
  \item We have $\iota'(e^n_{q+p'})^T=0$, from which we deduce
    $\iota'\p'=~\id{\overline{n}}$ by computing the matrix product.
  \item Considering the isomorphism
    $D^{1\times n}\simeq~D^{1\times\overline{n}}\oplus~D$, we have
    $\p=\p_1\p_2$, where $\p_1\in D^{n\times n}$ and
    $\p_2\in D^{n\times \overline{n}}$ are defined as follows:
    \medskip
    \begin{small}
      \[\p_1:=\begin{pmatrix}
      \id{\overline{n}}-F(e^{n}_{q+p'})^T\tilde{\ell} & 0\\
      0 & 1
      \end{pmatrix}\ \ \text{and}\ \
      \p_2:=\begin{pmatrix}
      \id{\overline{n}}\\
      \tilde{\ell}
      \end{pmatrix}.\]
    \end{small}
    \medskip
    From \eqref{equ:extend_Bezout_zero_part}, $\im(.\p_1)$ is included in
    $\ker(.F(e^{n}_{q+p'})^T)\oplus~D$ and the restriction of $.\p_2$ to
    the latter is injective: for $(u,x)\in\left(\ker(.F(e^{n}_{q+p'})^T)
    \oplus D\right)\cap\ker(.\p_2)$, we have $u+x\tilde{\ell}=0$, which
    gives $x=0$ and $u~=~0$ by \eqref{equ:extend_Bezout_zero_part}. Hence, we
    have $\ker(.\p)=\ker(.\p_1)$, that is
    $\ker\left(\id{\overline{n}}-.F(e^{n}_{q+p'})^T\tilde{\ell}\right)
    \oplus 0$. We conclude by showing $\ker
    \left(\id{\overline{n}}- .F(e^{n}_{q+p'})^T\tilde{\ell}\right)=D
    \tilde{\ell}$: the right to left inclusion is due to
    \eqref{equ:extend_Bezout_zero_part}, and the other one is due to the
    relation $x=\left(xF(e^n_{q+p'})^T\right)\tilde{\ell}$, for every $x$
    in $\ker\left(\id{\overline{n}}-.F(e^n_{q+p'})^T\tilde{\ell}\right)$.
  \item From \eqref{equ:extend_Bezout_zero_part}, we have 
    $D\ell UY_F\subseteq\ker(.\p')$. The converse inclusion is due to
    the relation $x=x_{q+p'}\ell UY_F$, for every $x\in\ker(.\p')$.
    Indeed, the first $q+\overline{p'}$ and the last $p+q'$ columns of
    $x$ and $x_{q+p'}\ell UY_F$ are equal since $x\in\ker(.\p')$ implies
    \medskip
    \begin{small}
      \[x\begin{pmatrix}
        \id{q+\overline{p'}} & 0\\
        0 & 0\\
        0 & \id{p+q'}
        \end{pmatrix}=x_{q+p'}\ell UY_F\begin{pmatrix}
        \id{q+\overline{p'}} & 0\\
        0 & 0\\
        0 & \id{p+q'}
        \end{pmatrix}.
      \]
    \end{small}
    \medskip
    Moreover, the $q+p'$-th colums of $x_{q+p'}\ell UY_F$ is computed
    by right multiplication by $(e^n_{q+p'})^T$ and is equal to
    $x_{q+p'}$ from \eqref{equ:extend_Bezout_zero_part}.
    \end{enumerate}
\end{pf}

\bigskip

\begin{thm}\label{thm:reduction_id}
  With the previous notations, we let
  \begin{center}
    \begin{tabular}{l l}
      $X_W:=X_F$, & $Y_W:=\iota UY_F\p'$,\\
      & \\
      $X_W^-:=X_F$, & $Y_W^-:=\iota'Y_F^-U^-\p$.
    \end{tabular}
  \end{center}
  \medskip
  The following diagram is exact and commutative:
  \begin{center}
    \begin{tikzcd}
      D^{1\times\overline{n}}\arrow[r, ".\tilde{L}"] \arrow[d, ".Y_W"', shift right=1ex] &
      D^{1\times m}\arrow[r, "\pi\oplus 0"] \arrow[d, ".X_W"', shift right=1ex]&
      M\arrow[r] \arrow[d, "f"', shift right=1ex]&
      0
      \\
      D^{1\times\overline{n}}\arrow[r, ".\tilde{L'}"] \arrow[u, ".Y_W^-"', shift right=1ex]&
      D^{1\times m}\arrow[r, "0\oplus\pi'"] \arrow[u, ".X_W^-"', shift right=1ex]&
      M'\arrow[r] \arrow[u, "f^-"', shift right=1ex]&
      0
    \end{tikzcd}
  \end{center}
  In particular, we have
  \[\tilde{L'}=Y_W^-\tilde{L}X_W.\]
\end{thm}

\begin{pf}
  We only have to show that the diagram is commutative.

  First, we show that $Y_W$ and $Y_W^-$ are inverse to each other.
  From Proposition~\ref{prop:exact_sequences_zero_part}, the lines of the
  following diagram are exact
  \medskip
  \begin{center}
    \begin{equation}\label{equ:diagram_reduction_zero}
      \begin{tikzcd}
        D\arrow[r, ".\ell"] \arrow[d, "\id{D}"']&
        D^{1\times n}\arrow[r, "\p"] \arrow[d, ".UY_F"', shift right=1ex]&
        D^{1\times\overline{n}}\arrow[r] \arrow[d, ".Y_W"', shift right=1ex]&
        0
        \\
        D\arrow[r, ".\ell UY_F"'] &
        D^{1\times n}\arrow[r, "\p'"'] \arrow[u, ".Y_F^-U^-"', shift right=1ex]&
        D^{1\times\overline{n}}\arrow[r] \arrow[u, ".Y_W^-"', shift right=1ex]&
        0
      \end{tikzcd}
    \end{equation}
  \end{center}
  Moreover, it is also commutative. Indeed, $\p Y_W$ is equal to
  $\p\iota U Y_F\p'$ and from~\ref{it:split_pi_zero_part} of
  Proposition~\ref{prop:exact_sequences_zero_part}, we have $\im(.\p\iota
    -\id{n})\subseteq\ker(.\p)$. By commutativity of the left rectangle
  and by exactness of the lines of \eqref{equ:diagram_reduction_zero}, we
  have $(\p\iota-\id{n})UY_F\p'=0$, so that $\p Y_W=UY_F\p'$. In the
  same manner, we show that $Y_F^-U^-\p=\p' Y_W^- $. By commutativity
  and exactness of \eqref{equ:diagram_reduction_zero} and from the
  equations $UY_FY_F^-U^-=Y_F^-U^-UY_F=\id{n}$, we get $Y_WY_W^-=~Y_W^-
  Y_W=\id{\overline{n}}$.

  Moreover, $Y_W\tilde{L}'=\tilde{L}X_F$ and $Y_W^-\tilde{L}=\tilde{L}'
  X_F^-$ follow from the following commutative diagram:
  \vspace{-0.4cm}
  \begin{small}
    \begin{center}
      \begin{equation}\label{equ:main_commutative_diagram_zero_part}
        \begin{tikzcd}
          D^{1\times\overline{n}}\arrow[d, "\tilde{L}"']\arrow[r, "\iota", shift right=-0.5ex] &
          D^{1\times n}\arrow[d, "L"]\arrow[r, "U", shift right=-0.5ex]
          \arrow[l, "\p", shift left=0.5ex] &
          D^{1\times n}\arrow[d, "L"]\arrow[r, "Y_F", shift right=-0.5ex] 
          \arrow[l, "U^-", shift right=-0.5ex] &
          D^{1\times n}\arrow[d, "L'"]\arrow[r, "\p'", shift right=-0.5ex] 
          \arrow[l, "Y_F^-", shift right=-0.5ex] &
          D^{1\times\overline{n}}\arrow[d, "\tilde{L}'"]
          \arrow[l, "\iota'", shift right=-0.5ex] 
          \\
          D^{1\times m}\arrow[r, "\id{m}"] &
          D^{1\times m}\arrow[r, "\id{m}", shift right=-0.5ex]
          &
          D^{1\times m}\arrow[r, "X_F", shift right=-0.5ex]
          &
          D^{1\times m}\arrow[r, "\id{m}", shift right=-0.5ex]
          \arrow[l, "X_F^-", shift right=-0.5ex] &
          D^{1\times m}
        \end{tikzcd}
      \end{equation}
    \end{center}
  \end{small}
  Indeed by computing the matrix products and
  from (\ref{equ:Fitting_eq}), we have the following relations:
  \begin{equation}
    \label{eq:formulas_reduction_zero_part}
    \begin{tabular}{l l l l}
      $\iota L=\tilde{L},$ & $UL=L,$ & $Y_FL'=LX_F$, & $\p'\tilde{L}'=L'$,\\
      & & \\
      $\iota'L'=\tilde{L}',$ & $Y_F^-L=L'X_F^-,$ & $U^-L=L$, & $\p\tilde{L}
      =L$.
    \end{tabular}
  \end{equation}
  More details are given in Section~\ref{sec:proofs_reduction_zero_part}.

\end{pf}

\subsection{Reduction of the identity bloc}

\bibliography{Source}

\end{document}

  
