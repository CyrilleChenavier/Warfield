%===============================================================================
% $Id: ifacconf.tex 19 2011-10-27 09:32:13Z jpuente $  
% Template for IFAC meeting papers
% Copyright (c) 2007-2008 International Federation of Automatic Control
%===============================================================================
\documentclass{ifacconf}

\usepackage{graphicx}      % include this line if your document contains figures
\usepackage{natbib}        % required for bibliography

\usepackage{amssymb,mathrsfs, amsmath}
\usepackage{tikz}
\usepackage{tikz-cd}

%newcommand
\newcommand\id[1]{\text{id}_{#1}}
\newcommand\sr{\text{sr}(D)}
%===============================================================================
\begin{document}
\begin{frontmatter}

\title{A constructive version of Warfield's Theorem} 
% Title, preferably not more than 10 words.


\author[First]{Cyrille Chenavier} 
\author[Second]{Alban Quadrat} 


\address[First]{Inria Lille - Nord Europe,
  Villeneuve d'Ascq, France (e-mail: cyrille.chenavier@inria.fr).}
\address[Second]{Inria Paris, Universit\'e Pierre et Marie Curie,  
   Paris, France (e-mail: alban.quadrat@inria.fr).}

\begin{abstract}                % Abstract of not more than 250 words.
\end{abstract}

\begin{keyword}

\end{keyword}

\end{frontmatter}
%===============================================================================

\section{Introduction}


\section{Module isomorphisms and equivalent matrices}

In this section, we recall the characterization of morphisms between
finitely presented left $D$-modules, as well as results of Fitting and
Warfield which rely isomorphic left $D$-modules to matrix conjugation.

\subsection{Effective version of Fitting's Theorem}

Consider two left $D$-modules $M$ and $M'$ with finite presentations:
\begin{equation}\label{equ:presentations}
  \begin{small}
    \begin{tikzcd}
      D^{1\times q}\arrow[r, ".R"] &
      D^{1\times p}\arrow[r, "\pi"] &
      M\arrow[r] &
      0, 
      \\
        D^{1\times q'}\arrow[r, ".R'"] &
        D^{1\times p'}\arrow[r, "\pi'"] &
        M'\arrow[r] &
        0, 
    \end{tikzcd}
  \end{small}
\end{equation}
namely, \emph{exact sequences} (see~\cite{Ro:09}), where
$R\in~ D^{q\times p}$, $(.R)(\mu)=\mu R$, for every $\mu\in D^{1\times q}$
and $\pi$ is the natural projection on $M=D^{1\times p}/(D^{1\times q}R)$
(similarly for $R'$ and $\pi'$).

From~\cite{Ro:09}, there exists $f\in\hom_D(M,M')$ if and only if there
exist matrices $P\in D^{p\times p'}$ and $Q\in D^{q\times q'}$ such that
$RP=QR'$ and
\begin{equation}\label{eq:def_of_f}
  \forall\lambda\in D^{1\times p},\ f(\pi(\lambda))=\pi'(\lambda\pi).
\end{equation}
Hence, the following diagram is exact and commutative:


\begin{center}
  \begin{small}
    \begin{tikzcd}
      D^{1\times q}\arrow[r, ".R"] \arrow[d, ".Q"'] &
      D^{1\times p}\arrow[r, "\pi"] \arrow[d, ".P"']&
      M\arrow[r] \arrow[d, "f"']&
      0
      \\
      D^{1\times q'}\arrow[r, ".R'"] &
      D^{1\times p'}\arrow[r, "\pi'"] &
      M'\arrow[r] &
      0
    \end{tikzcd}
  \end{small}
\end{center}

We let $n:=p+p'$ and $m:=q+p'+p+q'$. The two $m\times n$ matrices
\begin{equation}\label{equ:def_of_L}
  \begin{small}
    L:=\begin{pmatrix}
    R & 0 \\
    0 & \id{p'} \\
    0 & 0 \\
    0 & 0
    \end{pmatrix}\ \ \text{and}\ \
    L':=\begin{pmatrix}
    0 & 0 \\
    0 & 0 \\
    \id{p} & 0 \\
    0 & R'
    \end{pmatrix},
  \end{small}
\end{equation}
induce finite presentations:  $M\simeq D^{1\times n}/(D^{1\times m}L)$ 
and $M'\simeq D^{1\times n}/(D^{1\times m}L')$. In~\cite{ClQu:11}, an
effective version of a result due to~\cite{Fi:36} is given. If $f$ is an
isomorphism, then $L$ and $L'$ are equivalent: there exist $6$ matrices
$R_2\in D^{r\times q}$, $R'_2\in D^{r'\times q'}$,
$Z_2\in D^{p\times q}$, $Z_2\in D^{q\times r}$,
$Z'_2\in D^{q'\times r'}$, $Z\in D^{p\times q}$ and
$Z'\in D^{p'\times q'}$ and two invertible matrices of size $n$ and $m$
\begin{equation}\label{equ:def_of_XF}
  \begin{small}
    X_F:=\begin{pmatrix}
    \id{p} & P \\
    -P' & \id{p'}-P'P
  \end{pmatrix}\ \ \text{and}\ \
    Y_F:=\begin{pmatrix}
    \id{q} & 0 & R & Q \\
    0 & \id{p'} & -P' & Z' \\
    -Z & P & 0 & PZ'-ZQ \\
    -Q' & -R' & 0 & Z'_2R'_2
    \end{pmatrix},
  \end{small}
\end{equation}
with inverses
\begin{equation}\label{equ:def_of_XF-}
  \begin{small}
    X_F^-:=\begin{pmatrix}
    \id{p}-PP' & -P \\
    P' & \id{p'}
  \end{pmatrix}\ \ \text{and}\ \
    Y_F^-:=\begin{pmatrix}
    Z_2R_2& 0 & -R & -Q \\
    P'Z-Z'Q' & 0 & P' & -Z' \\
    Z & -P & \id{p} & 0 \\
    Q' & R' & 0 & \id{q'}
    \end{pmatrix},
  \end{small}
\end{equation}
such that
\begin{equation}\label{equ:Fitting_equiv}
  \begin{small}
    L'=Y_F^-LX_F.
  \end{small}
\end{equation}

\subsection{Warfield's Theorem}

A result due to~\cite{Wa:78} asserts that that the size of $0$ an id
blocs in (\ref{equ:def_of_L}) can be reduced, whereas the new matrices
are still equivalent. This result is based on the notion of
\emph{stable rank}. The definition of the latter requires to introduce
various notions that we present now.

A column vector $u:=(u_1\cdots u_k)^T\in D^{k\times 1}$ is called
\emph{unimodular} if there exists a line $v\in D^{1\times k}$ such that
$vu=1$. Moreover, $u$ is said to be \emph{stable} if there exist 
$d_1,\cdots,d_{k-1}\in D$ such that
$(u_1+d_1u_k\cdots u_{k-1}+d_{k-1}u_k)$ is unimodular. An integer $r$ is
said to be in the stable rank of $D$ if whenever $k>r$, every column
$u\in D^{k\times 1}$ is stable. The stable rank $\sr$ of $D$ is the
smallest integer in the stable rank of $D$.

Assume that the two matrices (\ref{equ:def_of_L}) are equivalent, then
Warfield's Theorem asserts that if there exist two integers $r$ and $s$
such that
\smallskip
\begin{equation}\label{equ:s_r}
  \left\{
  \begin{split}
    & s\leq\min(p+q',q+p'),\\
    & \sr\leq\max(p+q'-s,q+p'-s),\\
    &s\leq\min(p+q',q+p'), \\
    & \sr\leq\max(p+q'-s,q+p'-s),
  \end{split}
  \right.
\end{equation}
then the following $(m-r-s)\times (n-r)$ matrices are equivalent
\begin{equation}\label{equ:def_of_L_bar}
  \begin{small}
    \overline{L}:=\begin{pmatrix}
    R & 0 \\
    0 & \id{p'-r} \\
    0 & 0
    \end{pmatrix}\ \ \text{and}\ \
    \overline{L'}:=\begin{pmatrix}
    0 & 0 \\
    \id{p-r} & 0 \\
    0 & R'
    \end{pmatrix},
  \end{small}
\end{equation}
and induce finite presentations of $M$ and $M'$, respectively.

In the next section, we introduce a procedure which computes invertible
matrices $X_W$ and $Y_W$ such that
\begin{equation}\label{equ:Warfield_equiv}
  \overline{L'}=Y_W^-\overline{L}X_W.
\end{equation}

\bibliography{Source}

\end{document}
